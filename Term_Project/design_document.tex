\documentclass{report}
\begin{document}
\title{IMD3002 Term Project - Radiosity Rendering In Maya}
\author{Jake Cataford, Alex Winch}
\date{March 26th, 2014}
\maketitle

\begin{abstract}
This document outlines a plan to implement a radiosity renderer into Autodesk®
Maya. The implementation includes writing a c++ plugin for Maya using their SDK,
and wiring up a UI in Python to configure the renderer. The renderer will be
able to render a single frame to a file using Radiosity to simulate global
illumination for the scene. This document outlines the process of building such
a system, the objectives that should be accomplished as well as a background
surrounding the technology.
\end{abstract}
\tableofcontents
\chapter{Introduction}
This section of the document is to provide more information on the background of
the technology that we used in this implementation, As well as provide an overview
as to our teams particular approach and development environment, including any
overarching dependencies or major considerations that affect our team.

The renderer outlined in this document will be using a approach to simulate
global illumination called radiosity. Radiosity is a way to simulate light
bounces around the environment by interpreting the amount of albedo any given
patch (arbitrarily sized or other) can see in the scene. This method is then
recursed for [n] passes to increase the accuracy of the final product.

The renderer itself will be split into two different components: a python script
for allowing configuration in the Maya UI, and a C++ renderer dll/bundle plugin
to perform the actual rendering to a file. C++  is used for the rendering
process because it fits the Maya design patterns more idiomatically,
performs much faster, and is far more optimizable, than Python. The plugin will
be loaded in a similar way to the way Mental Ray is loaded, and used in
a similar fashion.
\chapter{Objectives}
This section outlines the objectives this project hopes to accomplish. Each
objective represents a complete task that is to be completed by the development
team in order for the project to succeed. Objectives do not provide
implementation details, only generalized statements representing success.
\begin{itemize}
\item This project should enable a user to render a scene from within Maya, without using any third party tools or dependencies.
\item This project should provide a way for the user to edit and configure the renderer to their personal preference.
\item This project should allow approximation of global ilumination through the radiosity technique.
\item The project source code should remain in a maintainable, modular state.
\item The renderer should run efficiently, algorithms within the renderer should have as low of a complexity as possible to reduce render time.
\item The project GUI should be implemented in as idiomatic of a way as possible. Following the design patterns that the Maya User Interface adheres to.
\item The renderer will render to a [single node] only to avoid spending too much time dealing with threading implementations.
\end{itemize}
\chapter{Method}
This section details the objectives in the previous section, and decorates them
with implementation details. The details in this section are not final, but
represent the plans for implementation based on our current knowledge of the
frameworks provided.

\section{The Renderer}
The renderer is a procedural c++ program that harnesses parameters from our GUI
and scene information from Maya, then compiles it into an image file by
traversing the data through a procedure. This program is not Object oriented
because the nature of a renderer is procedural, and there is nothing to gain
from having an application state. The renderer processes the data through
a variety of algorithms to render the scene geometry and complete the
rasterization pipeline. Below, you will find examples of the steps needed to
produce the final raster image.

The Technique requires reducing the scene's surfaces into patches, where each patch could
be represented by a struct as follows:

\begin{verbratim}
struct Patch:
  Vec3 emission;
  Vec3 reflectivity;
  Vec3 incidentLight;
  Vec3 excidentLight;
\end{verbratim}

Where each property is a color represented by a 3 Dimentional Vector (R,G,B).

To render the scene, consider this process:

\begin{itemize}
\item Load the scene.
\item Divide each surface in the scene into patches.
\item For each point light, raycast randomly around the scene and initialize the
  patches with some emission value.
\item For each area light, give all patches belonging to the light the emission
  value of the light.
\item For each ambient light, add that light value to the emission value of all
  patches.
\item For each patch, set patch emission to be the average albedo and color value of the pixels in the patch.
\item For each pass, loop through the patches in the scene and render the scene from the point of view of the patch. Record the sum of the render to the patch. Then calculate excident light via Incident light by reflectance, plus the previous emission value.
\end{itemize}

Directional lights require a different method of representing their influence on
scene light. We take the normal of the patch, compare it to the direction of the
directional lights in the scene, then set the render background (non-occluded
light) with an emission value of the dot product of the normal and light
direction. (Lambertian diffuse model).

\section{The GUI}

\chapter{Design Sketches}
This section is used to show off some of our GUI design concepts. This content
is due to change at any point if we discover limitations within our program.
\end{document}
